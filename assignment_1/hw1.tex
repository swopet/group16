\documentclass[letterpaper,10pt]{article}

\usepackage{graphicx}                                        
\usepackage{amssymb}                                         
\usepackage{amsmath}                                         
\usepackage{amsthm}                                          

\usepackage{alltt}                                           
\usepackage{float}
\usepackage{color}
\usepackage{url}

\usepackage{balance}
%\usepackage[TABBOTCAP, tight]{subfigure}
\usepackage{enumitem}
\usepackage{pstricks, pst-node}

\usepackage{geometry}
\geometry{textheight=8.5in, textwidth=6in}

%Allow sub titles
\usepackage{titling}
\newcommand{\subtitle}[1]{%
  \posttitle{%
    \par\end{center}
    \begin{center}\large#1\end{center}
    \vskip0.5em}%
}

%random comment

\newcommand{\cred}[1]{{\color{red}#1}}
\newcommand{\cblue}[1]{{\color{blue}#1}}

\usepackage{hyperref}
\usepackage{geometry}

\def\name{Trevor Swope, David Okubo, and Megan McCormick}

%pull in the necessary preamble matter for pygments output
\input{pygments.tex}

%% The following metadata will show up in the PDF properties
\hypersetup{
  colorlinks = true,
  urlcolor = black,
  pdfauthor = {\name},
  pdfkeywords = {cs444 ``operating systems'' files filesystem I/O},
  pdftitle = {CS 444 Homework 1 Write Up},
  pdfsubject = {CS 444 Homework 1},
  pdfpagemode = UseNone
}

%Set Up title page
\title{Project 1: Getting Acquainted}
\subtitle{CS 444, Spring 2018}
\author{Trevor Swope, David Okubo, and Megan McCormick}
\begin{document}

%Create Title page
\begin{titlingpage}
\maketitle 
\begin{abstract}
This paper details the progress we made in getting the qemu virtual machine set up on the server for the CS444 Operating Systems II class.
\end{abstract}
\end{titlingpage}
\section{Log of Commands}
\subsection{In Original SSH}
\begin{enumerate}
	\item cd /scratch/spring2018/16
	\item git init
	\item  cp /scratch/files/environment-setup-i586-poky-linux.csh environment-setp-i586-poky-linux.csh
	\item cp /scratch/files/core-image-lsb-sdk-qemux86.ext4 core-image-lsb-sdk-qemux86.ext4
  \item cp /scratch/files/bzImage-qemux86.bin bzImage-qemux86.bin
  \item vim README.md
  \item git add --all
  \item git commit -a
  \item git clone git://git.yoctoproject.org/linux-yocto-3.19.git
  \item cd linux-yocto-3.19
  \item git checkout tags/v3.19.2
  \item cp /scratch/files/config-3.19.2-yocto-standard .config
  \item make -j4 all
  \item cd ..
  \item qemu-system-i386 -gdb tcp::5516 -S -nographic -kernel /scratch/spring2018/16/linux-yocto-3.19/arch/x86/boot/bzImage -drive file=core-image-lsb         -sdk-qemux86.ext4,if=virtio -enable-kvm -net none -usb -localtime --no-reboot --append "root=/dev/vda rw console=ttyS0 debug"
\end{enumerate}

\subsection{IN ANOTHER SSH}
\begin{enumerate}
  \item cd /scratch/spring2018/16
  \item gdb -tui
  \item \$GDB
  \item target remote:5516
  \item Continue
\end{enumerate}

\subsection{IN VM}
\begin{enumerate}
  \item root
  \item shutdown now
\end{enumerate}

\section{Qemu Flags}
\begin{itemize}
  \item -gdb tcp::5516
    \subitem Wait for gdb connection on tcp::5516
  \item -S
    \subitem Do not start CPU at startup
  \item -nographic
    \subitem Disable graphical output so that QEMU is a simple command line application
  \item -kernel bzImage-quemux86.bin
    \subitem Use bzImage-quemux86.bin as kernel image
  \item -drive file=core-image-lsb-sdk-qemux86.ext4,if=virtio
    \subitem Define a new drive using the disk image "core-image-lsb-sdk-qemux86.ext4" connected with the virtio interface
  \item -enable-kvm
    \subitem Enable KVM full virtualization support
  \item -net none
    \subitem Indicates that no network devices should be configured
  \item -usb
    \subitem Enable the USB driver
  \item -localtime
    \subitem Set Real-time clock to start at local time
  \item --no-reboot
    \subitem Exit instead of rebooting
  \item --append "root=/dev/vda rw console=ttyS0 debug"
    \subitem Use "root=/dev/vda rw console=ttyS0 debug" as kernel command line
\end{itemize}

\section{Work Log}
\begin{itemize}
  \item 4/11/2018: Trevor set up kernel and VM
  \item 4/11/2018: David researched command flag meanings
  \item 4/11/2018: Megan started writeup
  \item 4/15/2018: Trevor finished and submitted writeup
\end{itemize}

\section{Version Control History}
\begin{tabular}{l l l}\textbf{Detail} & \textbf{Author} & \textbf{Description}\\\hline
\href{https://github.com/swopet/group16/commit/0ca32d3c9bb6e2ea87803bf2c1cedf9eced6cd88}{0ca32d3} & Trevor Swope & add readme\\\hline
\href{https://github.com/swopet/group16/commit/37d77a92bf5028137198c069a368434ea934e86b}{37d77a9} & Trevor Swope & add config stuff\\\hline
\href{https://github.com/swopet/group16/commit/5f7e1128983e84e7e80820817dc29c565faef752}{5f7e112} & Trevor Swope & Add command log\\\hline
\href{https://github.com/swopet/group16/commit/63a1aeb7a92fb6a74ef7c9a843fc9b95a72b4521}{63a1aeb} & Trevor Swope & edit .gitignore\\\hline
\href{https://github.com/swopet/group16/commit/db5d5f1394d2bae164a8aaffed3f9acb790090ed}{db5d5f1} & Trevor Swope & modify .gitignore\\\hline
\href{https://github.com/swopet/group16/commit/6d6cfcbee9bc5d8ec624fa8802d2a87c8de60433}{6d6cfcb} & Trevor Swope & add template and makefile\\\hline\end{tabular}


\nocite{*}
\bibliographystyle{ieeetr}
\bibliography{hw1}

%input the pygmentized output of mt19937ar.c, using a (hopefully) unique name
%this file only exists at compile time. Feel free to change that.
\end{document}
