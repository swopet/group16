\documentclass[letterpaper,10pt]{article}

\usepackage{graphicx}                                        
\usepackage{amssymb}                                         
\usepackage{amsmath}                                         
\usepackage{amsthm}                                          

\usepackage{alltt}                                           
\usepackage{float}
\usepackage{color}
\usepackage{url}

\usepackage{balance}
%\usepackage[TABBOTCAP, tight]{subfigure}
\usepackage{enumitem}
\usepackage{pstricks, pst-node}

\usepackage{geometry}
\geometry{textheight=8.5in, textwidth=6in}

%Allow sub titles
\usepackage{titling}
\newcommand{\subtitle}[1]{%
  \posttitle{%
    \par\end{center}
    \begin{center}\large#1\end{center}
    \vskip0.5em}%
}

%random comment

\newcommand{\cred}[1]{{\color{red}#1}}
\newcommand{\cblue}[1]{{\color{blue}#1}}

\usepackage{hyperref}
\usepackage{geometry}

\def\name{Trevor Swope, David Okubo, and Megan McCormick}

%pull in the necessary preamble matter for pygments output
\input{pygments.tex}

%% The following metadata will show up in the PDF properties
\hypersetup{
  colorlinks = true,
  urlcolor = black,
  pdfauthor = {\name},
  pdfkeywords = {cs444 ``operating systems'' files filesystem I/O},
  pdftitle = {CS 444 Homework 2 Write Up},
  pdfsubject = {CS 444 Homework 2},
  pdfpagemode = UseNone
}

%Set Up title page
\title{Project 2: I/O Elevators}
\subtitle{CS 444, Spring 2018}
\author{Trevor Swope, David Okubo, and Megan McCormick}
\begin{document}

%Create Title page
\begin{titlingpage}
\maketitle 
\begin{abstract}

\end{abstract}
\end{titlingpage}
\section{Questions}
\subsection{1. What do you think the main point of the assignment is?}

\subsection{2. How did you personally approach the problem? Design decisions, algorithms, etc.}

\subsection{3. How did you ensure your soltuion was correct? Testing details, for instance.}

\subsection{4. What did you learn?}

\subsection{5. How should the TA evaluate your work? Provide detailed steps to prove correctness.}

\section{Version Control History}
\begin{tabular}{l l l}\textbf{Detail} & \textbf{Author} & \textbf{Description}\\\hline
\href{https://github.com/swopet/group16/commit/0ca32d3c9bb6e2ea87803bf2c1cedf9eced6cd88}{0ca32d3} & Trevor Swope & add readme\\\hline
\href{https://github.com/swopet/group16/commit/37d77a92bf5028137198c069a368434ea934e86b}{37d77a9} & Trevor Swope & add config stuff\\\hline
\href{https://github.com/swopet/group16/commit/5f7e1128983e84e7e80820817dc29c565faef752}{5f7e112} & Trevor Swope & Add command log\\\hline
\href{https://github.com/swopet/group16/commit/63a1aeb7a92fb6a74ef7c9a843fc9b95a72b4521}{63a1aeb} & Trevor Swope & edit .gitignore\\\hline
\href{https://github.com/swopet/group16/commit/db5d5f1394d2bae164a8aaffed3f9acb790090ed}{db5d5f1} & Trevor Swope & modify .gitignore\\\hline
\href{https://github.com/swopet/group16/commit/6d6cfcbee9bc5d8ec624fa8802d2a87c8de60433}{6d6cfcb} & Trevor Swope & add template and makefile\\\hline\end{tabular}


\nocite{*}
\bibliographystyle{ieeetr}
\bibliography{hw1}

%input the pygmentized output of mt19937ar.c, using a (hopefully) unique name
%this file only exists at compile time. Feel free to change that.
\end{document}
